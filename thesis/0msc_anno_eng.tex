\chapter{Annotation}
The purpose of this research is to identify an impact of mobility on the Quality of Services (QoS) in the Wireless Sensor Networks (WSN).

The first part of the project involves an evaluation of the WSN infrastructure and the processes in it, through reviewing the already existing studies in this area.  At the end of the first part the model which gives a mathematical analysis of the main WSN processes and allows to calculate BERlink and BERroute , interference in the channel, also routes the length depending on the RB or NRBS switching techniques is being explained. The calculations obtained in this part can be applied to any Ad Hoc network.In the second part the overview of the WSN switching protocols is being made and the new SP-BER–based algorith is being proposed in order to improve the switching in the WSN system.

The final part involves the simulation of the WSN with the AODV and DSR switching protocol in NS-2 and compares its performance with following metrices PDF, average network delay, network throughput. The above-mentioned approach will allow to get the better overview of the WSN and have the insight view into the problems related to node mobility.

This master thesis is aligned with formal demands of Riga’s Technical University (RTU) for the Master Thesis and does not violate any copyrights. The thesis contains 32 figures, 5 tables, 25 equations and 41 sources of information. The total volume of the thesis is 77 page, including 3 appendices.