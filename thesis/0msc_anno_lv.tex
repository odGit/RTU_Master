\chapter{Anotācija}

Maģistra darba pētījuma mērķis ir izpētīt mobilitātes ietekmi uz bitu kļūdu intensitāti (BER) bezvadu sensoru tīklos (WSN), un uzlabot WSN maršrutā izvēli ar SP-BER balstīto maršruta izvēles shēmu.
Darba pirmā daļa ir veltīta WSN infrastruktūras un tīklā notiekošo procesu izpētei. Tiek sniegta galveno WSN processu matemātiskā analīze, nodrošinot iespēju aprēķināt BER savienojuma posmā, BER daudzposma maršrutā, interferences līmeni WSN tīklā, kā arī daudzposma maršruta garumu izmaņas atkarībā no izmantotā rezervēšanas (RB) vai bez-rezervēšanas (NRBS) komutācijas risinājuma. Aprakstītie matemātiskie aprēķini var būt pielietoti jebkurai no ekspromta tīkla topoloģijas analīzei. Noslēdzot šo daļu ar ekspromta tīkla  maršrutēšanas protokolu izpēti, noslēgumā aprakstot jauno SP-BER balstīto maršruta izvēles algoritmu.

Maģistra darba pēdējā nodaļā tiek veikta WSN simulācija izmantojot AODV un DSR protokolus NS-2 vidē salīdzinot tās efektivitāti mobilitātes apstākļos ar sekojošiem parametriem paketes piegādes sadale, paketes vidējo aizkavi tīklā un caurlaidspēju. Nodaļa sniedz ieskatu mobilos WSN tīklos, kā arī apskata ar mobilitāti saistītās problēmas un to risinājumus.

Maģistra darbs ir izstrādāts saskaņā ar Rīgas Tehniskās Universitātes (RTU) Maģistra darbu izstrādes prasībām un nekādā veidā nav pretrunā ar autortiesībām. Maģistra darbā ir iekļauts 32 attēls, 5 tabulas, 25 vienādojumi un 41 atsauces uz izmantotās literatūras avotiem. Maģistra darba kopējais apjoms, iekļaujot 3 pielikumus, sastāda 77 lapas puses.