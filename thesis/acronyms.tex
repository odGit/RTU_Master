\chapter{Saīsinājumu saraksts}
\begin{acronym}
%%%A%%%
\acro{Ad Hoc}   {\acroextra{\emph{ - Ad hoc - }}Ekspromta tīkls}
\acro{AODV}     {\acroextra{\emph{ - Ad Hoc On-Demand Distance Vector Routing - }}Ekspromta tīkla pēc pieprasījuma distances vektora protokols}
\acro{AWGN}	{\acroextra{\emph{ - Additive white Gaussian Noise - }}Aditīvs baltais Gausa troksnis}
%%%B%%%
\acro{BER}      {\acroextra{\emph{ - Bit Error Rate - }}Bitu kļūdu intensitāte}
\acro{BPSK}		{\acroextra{\emph{ - Binary Phase Shift Keying - }}Binārā fāzes manipulācija}
\acro{broadcast} {acroextra{ - }Apraide}
%%%C%%%
\acro{CBR}      {\acroextra{\emph{ - Constant Bit Rate - }}Konstants bitu ātrums}
\acro{CCDF}     {\acroextra{\emph{ - Complementary Cumulative Distribution Function - }} Absolūti nepārtraukts sadalījums}
\acro{CSMA/CA}	{\acroextra{\emph{ - Carrier sense, Multiple access/Collision avoidance - }}Nesēja jušanas un sadursmju nepieļaušanas daudzpiekļuve}
%%%D%%%
\acro{DSDV}     {\acroextra{\emph{ - Destination-Sequenced Distance-Vector - }}Galamērķa secīgā distances vektora protokols}
\acro{DSR}      {\acroextra{\emph{ - Dynamic Source Routing - }}Dinamiska avota maršrutēšana }
\acro{DSSS}	{\acroextra{\emph{ - Direct Sequence Spread Spectrum - }}Tiešās secības spektra paplašināšana}
%%%F%%%
\acro{FHSS}	{\acroextra{\emph{ - Frequency-Hopping Spread Spectrum - }}Frekvences lēkāšana}
%%%G%%%
\acro{GUID}	{\acroextra{\emph{ - Globally Unique Identifier  - }}Vispārēji unikāls identifikators}
%%%I%%%
\acro{IHWN}	{\acroextra{\emph{ - Intergrated Heterogenus Wireless Network - }}Integrēts heterogēnu bezvadu tīkls}
\acro{iid}      {\acroextra{\emph{ - Independent and identically distributed - }}Neatkarīgi un viendabīgi izkliedēti}
\acro{INI}      {\acroextra{\emph{ - Inter Node Interference - }}Mezglu savstarpējie traucējumi}
%%%L%%%
\acro{LAN}      {\acroextra{\emph{ - Local Area Network - }}Lokālais tīkls}
\acro{LR-WPAN}	{\acroextra{\emph{ - Low-Rate Wireless Personal Area Networks - }}Zemo-ātrumu bezvadu personālā apgabala tīkls}
\acro{LSRP}	{\acroextra{\emph{ - Link State Routing Protocol - }}Posma stāvokļa maršrutēšanas protokols}
%%%M%%%
\acro{MAC}      {\acroextra{\emph{ - Medium Access Control - }}Vides piekļuves vadības protokols}
\acro{MAN}	{\acroextra{\emph{ - Metropolitan Area Network - }}Pilsēttīkls}
\acro{MANET}    {\acroextra{\emph{ - Mobile Ad Hoc Network - }}Mobilais ekspromta tīkls}
\acro{MPR}      {\acroextra{\emph{ - Multi Point Relay - }}Vairākstāvokļu relejs}
\acro{multicast}{\acroextra{ - }Multiraide}
%%%N%%%
\acro{NRBS}     {\acroextra{\emph{ - Nonreservation-based switching - }}Bez-rezervēšanas komutācija}
\acro{NS-2}	{\acroextra{ - }Network Simulator 2}
%%%O%%%
\acro{OFDM}	{\acroextra{\emph{ - Orthogonal Frequency Division Multiplexing - }}Ortogonālā frekvenčdales multipleksēšana }
\acro{OLSR}     {\acroextra{\emph{ - Optimized link state routing protocol - }}Optimizēts posma stāvokļa maršrutēšanas protokols}
\acro{OMM}      {\acroextra{\emph{ - Obstacle Mobility Model - }}Šķēršļu mobilitātes modelis}
\acro{ONRBS}    {\acroextra{\emph{ - Opportunistic Nonereservation-based switching - }}Oportūnistiska bez rezervēšanas komutācija }
\acro{OSI}      {\acroextra{\emph{ - Open Systems Interconnection - }}Atvērto sistēmu starpsavienojums}
%%%P%%%
\acro{PER}	{\acroextra{\emph{ - Packer Error Rate - }}Pakešu kļūdu intensitāte}
\acro{PDF}      {\acroextra{\emph{ - Packet Delivery Fraction - }}Paketes piegādes sadale}
%%%R%%%
\acro{RBS}       {\acroextra{\emph{ - Reservation-based switching - }}Rezervēšanas komutācija }
\acro{RESGO}    {\acroextra{\emph{ - Reserve-and-Go MAC - }}Rezervē un strādā MAC protokols}
\acro{RREP}     {\acroextra{\emph{ - Route Reply - }}Maršruta atbildes pakete}
\acro{RERR}     {\acroextra{\emph{ - Route Request ERROR - }}Maršruta pieprasījuma ERROR pakete}
\acro{RREQ}     {\acroextra{\emph{ - Route Request - }}Maršruta pieprasījuma pakete}
\acro{RDMM}     {\acroextra{\emph{ - Random Direction Mobility model - }}Gadījuma virzienu mobilitātes modelis}
\acro{RWM}      {\acroextra{\emph{ - Random Walk Mobility model - }}Gadījuma iešanas mobilitātes modelis}
\acro{RWMM}     {\acroextra{\emph{ - Random Waypoint Mobility model - }}Gadījuma maršrutpunktu mobilitātes modelis}
%%%S%%%
\acro{SNR}      {\acroextra{\emph{ - Signal to Noise Ratio - }}Signāla un trokšņa attiecība}
\acro{SIR}      {\acroextra{\emph{ - Signal to Interference Ratio - }}Signāla un  interferences attiecība}
%%%%T%%%
\acro{TC}	{\acroextra{\emph{ - Topology Constrol Packet - }}Topoloģijas uzraudzības ziņojums}
%%%U%%%
\acro{unicast}	{\acroextra{ - }Uniraide}
\acro{UWB}		{\acroextra{\emph{ - Ultra Wide Band - }}Ultraplatjosla}
%%%V%%%
\acro{VANET}    {\acroextra{\emph{ - Vehicular Ad-Hoc Network - }}Transportlīdzekļu sensoru tīkls}
%%%Q%%%
\acro{QoS}      {\acroextra{\emph{ - Quality of Service - }}Servisa kvalitāte}
%%%W%%%
\acro{Wi-Fi}    {\acroextra{\emph{ - Wireless Fidelity - }}Bezvadu precizitāte tīkls}
\acro{WLAN}     {\acroextra{\emph{ - Wireless Local Area Network - }}Bezvadu lokālais tīkls}
\acro{WMN}      {\acroextra{\emph{ - Wireless Mesh Network - }}Bezvadu režģtīkls }
\acro{WPAN}     {\acroextra{\emph{ - Wireless Personal Area Network - }}Bezvadu personālā apgabala tīkls}
\acro{WSN}      {\acroextra{\emph{ - Wireless Sensor Network - }}Bezvadu sensoru tīkls}
\end{acronym}
%\acrodef works outside the acronym environment