
%to use \gls{ber_link}
\newglossaryentry{ber_link} {name=$BER_{link}$,description={BER līmenis galā saikni starp divām kaimiņu mezgliem}}
\newglossaryentry{p_therm}  {name=$P_{therm}$,description={Termiskā trokšņa jauda, [dBm]}}
\newglossaryentry{p_t}      {name=$P_{t}$,description={pārraides jauda, [dBm]}}
\newglossaryentry{p_imin}   {name=$P_{i_{min}}$, description={minimāla pārraides jauda, [dBm]}}
\newglossaryentry{r_link}   {name=$r_{link}$,description={attālums starp uztvērējmezglu un raidītājmezglu, [m]}}
\newglossaryentry{v_max}    {name=$\upsilon_{max}$,description = {maksimālais starpmezglu ātrums, [m/s]}}
\newglossaryentry{v}        {name=$\upsilon$,description={mezgla ātrums, kas vienmērīgi izkliedēts $(0, \upsilon_{max}]$ intervālā}}
\newglossaryentry{theta}    {name=$\theta$,description={leņķiskais ātrums, kas vienmērīgi izkliedēts $[0, 2\pi]$ intervālā}}
\newglossaryentry{lambda}   {name=$\lambda$,description={lielums kas raksturo pakešu ražošanas ātrumu, $[pck/s]$}}
\newglossaryentry{L}        {name=$L$, description={paketes garums, $[b/pck]$}}
\newglossaryentry{r_b}      {name=$R_{b}$, description={datu pārsūtīšanas ātrums, $[kb/s]$}}
\newglossaryentry{rho}      {name=$\rho_{s}$, description={mezglu telpas blīvums, $[\frac{1}{m^2}]$}}


