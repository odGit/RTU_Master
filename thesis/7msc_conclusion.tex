\chapter{Secinājumi}\label{sec:secin}
% raksturo materiālu (datu) vākšanā un analīzē pielietoto pētniecisko metodi, apkopo maģistra darba argumentāciju un teorētiskos un praktiskos secinājumus.
% Darba nobeigumā autors lakoniski definē secinājumus, darba rezultātus un iespējamās rekomendācijas. Pēc maģistra darba teksta, seko secinājumi (tēzes) īsā, lakoniskā veidā, rindkopu formā. Tās numerācijas secībā
%3. izvēlētās pētniecības metodes un objektus
Tuvāk iepazīstoties ar mobiliem bezvadu sensoru tīkliem (WSN) kļūst skaidrs, ka tīkla veiktspēja un servisa kvalitāte (QoS) tajā ir atkarīgi no daudziem faktoriem. Kuru starpā vissvarīgākie ir mobilitāte, mezglu telpiskais blīvums un maršrutēšanas protokola veiktspēja. Pārsvarā mobilie WSN tīkli tiek pētīti pie ideāliem apstākļiem izdalot tājos vienu vienīgu pētāmo parametru. Šajā darbā tika izmantota semi-analītiskā pieeja lai pētītu saistības starp mezglu mobilitāti, maršrutēšanas protokolu un fiziskā slāņa īpašību ietekmi uz QoS līmeni daudzposmu maršrutā.

Bezvadu sensoru tīkli ir diezgan jauna tehnoloģija, un līdz ar to trūkst reālo datu par tajā notiekošiem procesiem, kā arī pētīšanas process tiek apgrūtināts ar ļoti plašu pielietošanas spektru. Pēc šī darba izstrādes viennozīmīgi varu apgalvot, ka lai nodrošinātu pētniecisku progresu šinī nozarē ir nepieciešams: 1) izveidot pamatnostādnes un ierobežot pielietošanas apgabalu 2) būvēt un  pētīt reālus tīklus, nevis balstīt pētījumus tikai uz datorsimulācijas rezultātiem. Nodaļā \ref{sec:tris} apkopoti teorētiskie aspekti var tikt pielietoti, kā sākum pozīcija WSN fiziskā slāņa parametru aprēķiniem.

Darba gaitā tika piedāvāta maršruta izvēles algoritma optimizācija (\seename ~\ref{sec:BERSP} sadaļu), piedāvātais algoritms ir kompromiss starp BER līmeni un posmu skaitu maršrutā. Tas izvēlas maršrutu ar vismazāko posmu skaitu no iespējamiem maršrutiem kuros $BER_{route}$ ir zemāks par $BER_{app}$. Kur  $BER_{app}$ līmeni nosāka lietojumprogramma vai tīklā administrators. Gadījumā, kad visu iespējamo maršrutu līmenis $BER_{route}$ ir augstāks par $BER_{app}$ BER-SP algoritms darbosies, kā BER-bāzēts algoritms un izvēlēsies maršrutu ar viszemāko $BER_{route}$. Un gadījumā, kad pieejamu maršrutu $BER_{route}$ ir zemāks par $BER_{app}$, BER-SP algoritms darbosies pēc SP algoritma principā, izvēloties visīsāko maršrutu jo visiem maršrutiem $BER_{route}$ ir zemāk par $BER_{app}$. Ar nelielu simulāciju MATLABā (\seename ~\figurename. ~\ref{fig:SPBER_ber}) tika pierādīts ka šim paņēmienam ir acīmredzama priekšrocība salīdzinājumā ar īsākā ceļa (short-path) un BER-bāzes algoritmu. Lai pilnībā novērtētu šo algoritmu ir nepieciešami papildus pētījumi NS-2 vidē.

Attēlā \ref{fig:bliv} parādīta likumsakarība starp BER līmeni daudzposmu maršrutā un mezglu telpisko blīvumu tīklā. Pie mezglu telpiskā blīvuma zemāk par $5\times10^{-4}$ BER līmenis daudzposmu maršruta beigās sasniedz 1. Kas ir izskaidrojams ar pārāk lielu distanci starp raidītāju un uztvērēju. Ja aprēķiniem tiek izmantot AWGN kanāls tad BER no $5\times10^{-4}$ tas strauji tiecas uz 0. Kas neatbilst reālajai situācijai, jo tuvāk ir raidošie mezgli jo augstāka ir mezglu savstarpējā interference. Gadījumā kad datu pārraide tīklā tiek ietekmēta ar augstu interferenci, maršrutēšanas protokols nevar pozitīvi ietekmēt tīkla veiktspēju \cite{qoS_mobility}. Līdz ar to NS-2 simulācijai tika izvēlēti divi telpiski blīvumi $\rho_{1500\times300} = 1.13\times10^{-4}$ un $\rho_{500\times300} = 3.4\times10^{-4}$, [$\frac{1}{m^{2}}$].

Pēc darba izstrādes ir grūti viennozīmīgi pateikt vai mobilitātei ir pozitīvā vai negatīva ietekme, jo tas, ko esmu konstatējusi ar veiktajām simulācijām un aprēķiniem  ir tas, ka ir jāpēta parametru kombinācijas: mezglu kustības ātrums, telpiskais blīvums, datu pārraides ātrums un izmantotais pakešu garums. Kā arī mobilitātes modelim ir ļoti liela ietekme uz rezultātiem. Attēlā ~\ref{fig:pause} ir pāradīti rezultāti AODV protokola veiktspēja pie RWMM dažādiem pauzes garumiem. Kad pauzes laiks ir intervālā no 30 līdz 120 AODV protokolā veiktspēja pasliktinās par ~10\% salīdzinājumā kad tas ir 0. Tas ir jāņem vērā novērtējot mobilitāti. Taču nav iespējams aprēķināt visas parametru kombinācijas un to lielumu variācijas. Tātad, var secināt, ka ir svarīgi zināt sistēmas prasības, lai varētu noteikt mobilitātes ietekmi.

Šai darbā izvēlētajai sistēmai (\seename \ref{sec:petPar} sadaļu) pie maksimāljiem mezglu kustības ātrumiem 5 un 10 [m/s] un izmantojot AODV maršrutēšanas protokolu mobilitātes ietekme ir bijusi pozitīva. \acf{PDF} (\seename ~\figurename~\ref{fig:pdfspeed}) sasniedz 0.8 un ir novērojama viszemākā vidējā pakešu aizkave < 10 sek. (\seename ~\figurename ~\ref{fig:avgDspeed}) un tīkla caurlaidspēja pārsniedz 120 [Kbit/s] (\seename ~\figurename ~\ref{fig:avgT}). Balstoties uz iegūtiem rezultātiem, var viennozīmīgi apgalvot, ka AODV protokola veiktspēja, it sevišķi pie zemā telpiska blīvuma, ir daudz labāk nekā DSR (\seename ~\figurename ~\ref{fig:avgDspeed}, ~\ref{fig:pdfspeed}, ~\ref{fig:avgT}). AODV spēj ātrāk pielāgot maršrutu manīgas topoloģijas apstākļos ar mazāku ietekmi uz caurlaidspēju.

Vēl jāpiemin, ka NS-2 datorsimulācija ir teicams līdzeklis, lai varētu simulēt tīkla veiktspēju pie dotiem parametriem. Tas piedāvā iespējo izmantot jau iebūvētos maršrutēšanas protokola, MAC vai fiziskā slāņa modeļus, kā arī iespēju izveidot uz tā bāzes savus modeļus. Šajā darbā tika izmantoti iebūvēti NS-2 komponenti, kas bija modificēti lai atbilstu šī darba mērķiem. Šī darba praktiskā daļa varētu būt noderīga studentiem iepazīstot mobilo bezvadu sensoru tīklu (WSN) darbību. Kā arī jāmin tas, ka NS-2 simulācijas rezultāti ir pieejami ne vien teksta failā (\texttt{.tr}), bet arī tie ir pieejama katrā scenārija vizualizācijā (\texttt{*.nam}), kas palīdz labāk izprast tīklā notiekošos procesus.

Šis darbs var tikt turpināts sekojošos virzienos. Pirmkārt, izmantot ZigBee/802.15 MAC, lai novērtētu kā dažādas tīkla topoloģijas darbojas mobilitātes apstākļos. Otrkārt, NS-2 vidē uz AODV bāzēs izveidot BER-SP algoritmu un salīdzināt to veiktspēju ar oriģinālo AODV. Treškārt, MATLABā Simulink izveidot WSN kanāla modeli un pētīt tajos notiekošos procesus ar dažādiem antenu parametriem un modulāciju veidiem.

Darbā izvirzītie uzdevumi ir izpildīti un varu uzskatīt, ka maģistra darba mērķis ir sasniegts.

% =====================================================
%  The proposed framework can be used in conjunction with any mobility model, provided that a suitable statistical description is available. Our results show that the use of ONRBS allows supporting, at the expense of heavier control traffic, a higher mobility level than the use of RBS. We have also shown that the larger the traffic load (and, consequently, the interference), the lower is the impact of the routing (or switching) strategy (i.e., RBS versus ONRBS) on the network performance. Two mobility models, namely DP and DNP, have been considered. Our results show that, in RBS-based ad hoc wireless networks, DNP mobility supports a better performance than DP mobility, since frequent changes of directions average out, forcing the nodes to move around their original positions, rather than moving far away and, therefore, disrupting connectivity.
